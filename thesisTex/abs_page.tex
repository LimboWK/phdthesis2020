\documentclass[10.5pt,twoside]{mitthesis}
\usepackage{lgrind}
\usepackage{graphicx}
\usepackage{amsmath}
\usepackage{booktabs}
\usepackage{float}
\usepackage{ amssymb }

\RequirePackage{doi}
\usepackage{hyperref}
\usepackage{url}
\usepackage{graphicx}
\usepackage[graphicx]{realboxes}
\usepackage{rotating}
\usepackage{CJKutf8}

\usepackage[labelformat=simple]{subcaption}
\renewcommand\thesubfigure{\alph{subfigure})}

\usepackage{comment}
\usepackage{natbib}
\setcitestyle{numbers}
\usepackage[compat=1.1.0]{tikz-feynman}

\usepackage{ifthen}
\pagestyle{plain}
\graphicspath{{figures/}}

\DeclareUnicodeCharacter{00A0}{ }
%% This bit allows you to either specify only the files which you wish to
%% process, or `all' to process all files which you \include.
%% Krishna Sethuraman (1990).
%
%\typein [\files]{Enter file names to process, (chap1,chap2 ...), or `all' to process all files:}
%\def\all{all}
%\ifx\files\all \typeout{Including all files.} \else \typeout{Including only \files.} \includeonly{\files} \fi

\begin{document}

\begin{CJK}{UTF8}{min}
	\large
	\centering
	論文の内容の要旨\\
\end{CJK}

\begin{center}
	\large
	Time Dependent Charge-Parity Violation in $B^0 \to K^0_S K^0_S K^0_S $ in Belle II early operation\\
\end{center}
\begin{CJK}{UTF8}{min}
	\large
	\centering
	(Belle II 初期データを使った$B^0 \to K_S^0  K_S^0  K_S^0$ 崩壊の時間に依存する荷電・パリティ非保存の研究)\\
	\vspace{1cm}
\end{CJK}
\begin{CJK}{UTF8}{min}
	\large
	\centering
	氏 名 \space \space \space \space  万 琨\\
\end{CJK}
\begin{center}
	\large
	Wan Kun\\
\end{center}
\vspace{1cm}
The Belle II experiment is a next-generation super $B$-factory experiment. The targeted instantaneous luminosity is 
$8 \times 10^{35}~ \text{cm}^{-2}\text{s}^{-1}$ and the expected integrated luminosity is 50 ab$^{-1}$ by 2030 with the majority of data collected at the $\Upsilon(4S)$ resonance using SuperKEKB accelerator.

The thesis is based on the time-dependent $\it{CP}$ violation study of $B^0 \to K_S^0 K_S^0 K_S^0$ decay to precisely measure the $\it{CP}$ parameters $\mathcal{S}$ and $\mathcal{A}$ in penguin-dominated $b \to s$ transition, which is sensitive to New Physics effects. Such a precise measurement mainly depends on determination of the distance between two vertices of two neutral $B$ mesons. The blind analysis and fit by a unbinned maximum likelihood method are performed using about 62.8 fb$^{-1}$ recorded experiment data from Belle II detector 2019 and 2020 (spring and summer) operation. The measurement results: $\mathcal{S}= - \text{sin}(2\phi_1) = -0.82 \pm 0.85~(\text{stat}) \pm 0.07 ~(\text{syst})$ and $\mathcal{A}= -0.21 \pm 0.28 ~ (\text{stat}) \pm 0.06 ~ (\text{syst})$ are obtained. The result is dominated by the statistical uncertainty and currently consistent with the Standard Model, as well as the previous results in the Belle and BaBar. 

At the 50 ab$^{-1}$ future Belle II full luminosity, we estimate the total uncertainty of $\mathcal{S}$ to be $0.040\sim 0.048$ in $B^0 \to K_S^0  K_S^0  K_S^0$. This estimation is a preliminary result based on the limited data at present and conservative assumptions on the reduction of the uncertainties, which is subject to change in future. Overall, such total uncertainty can provide a much better probe for searching the New Physics effects in future.

\end{document}
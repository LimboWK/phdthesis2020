\chapter{Discussions, conclusions and prospect}


 %The Belle II experiment is crucial in these channels because of the cleaner background environment and better sensitivity compared with LHCb.
\section{Discussions.}
Based on the current status of the analysis, several important topics need to be discussed. In this section, we discuss the statistical and systematic uncertainty improvements in future based on the current results, as well as the importance of \textit{KsFinder} in this analysis. 
\subsection{Statistical uncertainty in future}
The precision on $\mathcal{S}$ requires the large luminosity of data as shown in Figure \ref{fig:sensitivity}, which includes both statistical and systematic uncertainties. With 50 ab$^{-1}$ luminosity from the full Belle II data sample in future, the statistical uncertainty is expected to be largely reduced. The $\it{CP}$ fit on the MC sample with different amount of events used reflects that the statistical uncertainty is reduced proportionally around factor of $\frac{1}{\sqrt{N}}$, where $N$ is the events used in $\it{CP}$ fit. Also, due to the flavor tagging accuracy limitation, the expected statistics of correctly tagged events could be increased, while at this moment, there is no clear sign of large improvement on the reduction of wrong tagging fraction. The reduction of statistical uncertainty is assumed to come from the increased data sample with current reconstruction efficiency, which is shown in Figure \ref{fig:stats_future}. At 50 ab$^{-1}$ luminosity, the statistical uncertainty is estimated to be $\sim 0.03$.

\begin{figure}[htpb]
	\centering
	\includegraphics[width=0.7\linewidth]{stats_b2}
	\caption{Statistical uncertainty of $\mathcal{S}$ extrapolation based on the current result of $B^0 \to K_S^0  K_S^0  K_S^0$ in Belle II, where the orange is the current value and the green is the Belle result at 0.711 ab$^{-1}$ $\Upsilon$(4S) data.}
	\label{fig:stats_future}
\end{figure}

From Table \ref{tab:b0stats}, the current $B^0$ reconstruction efficiency is 34\% which could be further optimized mainly by improving $K_S^0$ reconstruction efficiency. As discussed in section 3, the reconstruction efficiency and vertexing quality become worse for long-flight $K_S^0$ particles. This is mainly due to the limitation of CDC-only tracking and the hit filters on the SVD layers. The current Belle II track finding algorithm rises a requirement for SVD hits that at least two or more SVD hits have to be associated from a same track so that they can be used together to form a track. If a $K_S^0$ decays outside of layer 5 at 10.4 cm, even though the daughter tracks could pass the SVD layer 6, they are much likely to become two CDC-only tracks unless they pass the overlapped regions at the edges of SVD layers. 
This effect is shown in Figure \ref{fig:svd-r-xx}, where \textit{SVD00} type $K_S^0$ starts to appear at SVD layer 5 instead of layer 6. The requirement of the SVD hits in the current tracking algorithm is needed to suppress the beam background and SVD noise strips that create a large fraction of random single hits. Thus, the actual sensitive volume of SVD is reduced and the $K_S^0$ reconstruction efficiency is negatively affected. 
In future, the improvement of our tracking algorithm is expected to remove this requirement while still be able to effectively reduce single-hit background. 



In general, the expected $B^0$ signal yield can be improved in future and help to further reduce the statistical uncertainty. From Figure \ref{fig:stats_future}, the extrapolated statistical uncertainty at $0.711$ ab$^{-1}$ is comparable with the Belle result. 
When the integrated luminosity reaches about 9 ab$^{-1}$, the statistical uncertainty is reduced to $\sim 0.072$ which is equivalent to the current systematic uncertainty. At 50 ab$^{-1}$ integrated luminosity, the major contribution will be systematic uncertainty if no improvement is assumed. We take the extrapolated statistical uncertainty of $\sim 0.030$ at 50 ab$^{-1}$ as a conservative value for estimating the total uncertainty later.



\begin{figure}[htpb]
	\centering
	\includegraphics[width=0.7\linewidth]{ks-r-svdxx}
	\caption{The $K_S^0$ are categorized into 4 types. \textit{SVD11} (\textit{SVD00}) are the ones whose daughter pion tracks have non-zero (zero) SVD hits. \textit{SVD10} (\textit{SVD01}) stands for the ones whose only positive (negative) charged pion track contains SVD hits.
		 The distribution is the $K_S^0$ flight length on $x-y$ plane for each category of $K_S^0$, where \textit{SVD00} type $K_S^0$ start to appear at about SVD layer 5.}
	\label{fig:svd-r-xx}
\end{figure}

\subsection{Systematic uncertainty in future}
As discussed in the section 1, the NP effects that potentially contributes to the $B^0\to \phi K_S^0$ could also affect $B^0 \to K_S^0  K_S^0  K_S^0$. In the current SM correction, the QCD factorization (QCDF) scan approach suggests the expected upper-limit for $\Delta \mathcal{S}$ is about 0.05~\cite{b2book}. In this scale, it is important to evaluate the reducible and irreducible systematic uncertainties in $B^0 \to K_S^0  K_S^0  K_S^0$ for future data collection based on the current measurement result. If no improvement on systematic uncertainty is expected from the current evaluation, it is still challenging to validate the evidence of the NO effects against the small theoretical predictions with the full Belle II luminosity in future.
%QCDF generally predicts definite or preferred signs of the $\Delta \mathcal{S}$ shift, which implies a definite pattern of shifts to be compared with data\cite{b2book}. At present thereis no significant tension between these predictions and data. 

The irreducible sources mainly refer to the ones that are not improved with increased luminosity, such as the irreducible vertexing related sources and tag-side interference~\cite{b2book}. The reducible sources are mainly the signal $\Delta t$ shape , the signal fraction, the background $\Delta t$ shape, the flavor tagging, and the fit bias. As an initial study in early operation with very low statistics, it is assumed that the reducible sources are achieved by the increased future luminosity where the reduction of uncertainties of the parameters are scaled by the fraction of squared-root of increased events.

In the Belle II original prospect, the new PXD detector would be contributive to reduce about 50\% of the systematic uncertainties caused by the vertexing resolution. However, the PXD is not fully installed at the current Belle II  and the vertex reconstruction in this analysis doesn't include IP constraint options. The PXD is scheduled to be fully installed in 2023. Hence the conservative scenario that systematic uncertainties do not benefit from the improved vertexing quality is assumed.

For signal $\Delta t$ shape, we assume that the $\it{CP}$-side resolution function parameters can be optimized by increasing \textit{signal MC}. The current contribution from $\it{CP}$-side resolution parameters is $\sim 0.030$, which can be reduced to $0.015$ if 50\% reduction is achieved.
For the tag-side parameters, the current contribution is $\sim 0.016$. We expect the contributions from these parameters to be reduced to $0.003$ at 50 ab$^{-1}$ luminosity. In Table  \ref{tab:sig_shape}, the improvement where both $\it{CP}$- and tag-sides uncertainties are reduced is calculated to be $0.015$ while a conservative case that only tag-side is improved is also considered, which is $\sim0.030$.
\begin{table}[H]
		\centering
		\caption{ Signal $\Delta t$ shape systematic uncertainties of $\mathcal{S}$ expected at 50 ab$^{-1}$. The second and third columns are the expected reduced systematic uncertainties for both $\it{CP}$/tag-side improvements or only tag-side improvement.}
		\label{tab:sig_shape}
		\begin{tabular}{c| c| c }
			\hline
			Luminosity (50 ab$^{-1}$) & both ($\it{CP}$/tag) improved & only tag-side improved \\
			\hline
			signal $\Delta t$ shape &  $\sim0.015$ & $\sim0.030$\\
			\hline
		\end{tabular}
\end{table}
For the signal fraction contribution which is the largest source at this moment, it mainly suffers from the very low statistics in data that causes the inaccurate modeling of signal shape parameters. Therefore the uncertainties of signal fraction parameters are expected to be reduced quickly with the increased data collection in future. Using 1 ab$^{-1}$ \textit{generic MC}, the combined contribution on $\mathcal{S}$ uncertainty by floating $\pm 1 \sigma$ for each parameter is calculated to be $\sim 0.013$. This estimation is close to the observed systematic uncertainty in the Belle data which is  $\sim0.015$~\cite{kang2020measurement} and the Belle II data scaled value $\sim 0.011$. In 50 ab$^{-1}$ Belle II data, we assume this source can be reduced to $\sim 0.002$ as shown in Table \ref{tab:sig_f}.

\begin{table}[htpb]
	\centering
	\caption{ Signal fraction systematic uncertainties of $\mathcal{S}$ expected at 50 ab$^{-1}$.}
	\label{tab:sig_f}
	\begin{tabular}{c| c}
		\hline
		Luminosity (50 ab$^{-1}$) & Improved uncertainty \\
		\hline
		Signal fraction &  $\sim0.002$ \\
		\hline
	\end{tabular}
\end{table}

For the background $\Delta t$ shape, the systematic uncertainty is reduced by factor $\sqrt{50/0.063}$ to be $\sim 0.001$ as listed in Table \ref{tab:bkg_shape}.

\begin{table}[htpb]
	\centering
	\caption{ Background $\Delta t$ shape systematic uncertainties of $\mathcal{S}$ expected at 50 ab$^{-1}$.}
	\label{tab:bkg_shape}
	\begin{tabular}{c| c}
		\hline
		Luminosity (50 ab$^{-1}$) & Improved uncertainty \\
		\hline
		Background $\Delta t$ shape &  $\sim0.001$ \\
		\hline
	\end{tabular}
\end{table}

For the contributions of wrong tag fraction, the uncertainties of $w$ in each $r$-bin is expected to be reduced with more tagging control samples.  The expected uncertainties at 50 ab$^{-1}$ Belle II data is assumed to be $\sim 7$ times smaller than those from the Belle result~\cite{kang2020measurement}, which is about $\sim0.002$ as listed in Table \ref{tab:wtag_50ab}.

\begin{table}[htpb]
	\centering
	\caption{ Wrong tag fraction systematic uncertainties of $\mathcal{S}$ expected at 50 ab$^{-1}$.}
	\label{tab:wtag_50ab}
	\begin{tabular}{c| c}
		\hline
		Luminosity (50 ab$^{-1}$) & Improved uncertainty \\
		\hline
		wrong tag fraction &  $\sim0.002$\\
		\hline
	\end{tabular}
\end{table}

For the fit bias contribution, currently the values are taken by the statistical fit error using 300000 events in \textit{signal MC}. If MC sample used in future could be at least 100 times more than one million, then the fit error is possible to be smaller than input-output difference. From the current MC production plan of the Belle II, the  \textit{signal MC} sample recommended by MC production group is typically in a range of several millions. So the foreseen systematic uncertainty is still going to be the fit error, where we take a 50\% reduction of the value $\sim0.001$ from Tabel \ref{tab:fitbias} as an estimation, listed in Table \ref{tab:fitbias_full}.

\begin{table}[htpb]
	\centering
	\caption{ Fit bias systematic uncertainties of $\mathcal{S}$ expected at 50 ab$^{-1}$.}
	\label{tab:fitbias_full}
	\begin{tabular}{c| c}
		\hline
		Luminosity (50 ab$^{-1}$) & Improved uncertainty \\
		\hline
		fit bias &  $\sim 0.005$ \\
		\hline
	\end{tabular}
\end{table}

Concluded from the above discussion, the reducible systematic uncertainties by using increased MC and data in the full Belle II luminosity are estimated and summarized in Table \ref{tab:reducedsys}. It is clear that the dominated contribution in future Belle II data for systematic uncertainty is the $\it{CP}$ side resolution, indicating the finer study on $\it{CP}$ side resolution model for no IP-originated tracks is important.

The impact of using \textit{KsFinder} receives contribution from the data-MC mismatch, which can be be improved by a better data-MC consistency. For physics parameters $\Delta m_d$ and $\tau_{B^0}$, the uncertainties could be reduced by the improved physics input. The vertex reconstruction options are not contributing much in this analysis mostly because they are partially reflected by the signal $\Delta t$ shape contributions. Using proper IP constraint and tighter vertex cuts, the signal $\Delta t$ shape will contribute less and vertex reconstruction could contribute more in future. The tag-side interference contribution is referenced to be $\sim0.001$ as a small source from the Belle study~\cite{yosuke2011measurement}.   The current estimations from \textit{KsFinder} ($\sim0.005$), physics parameters($\sim$0.007), the vertex reconstruction($\sim$0.019) and tag-side interference($\sim0.001$) are assumed to be unchanged in the 50 ab$^{-1}$ Belle II luminosity to have a conservative expectation on $\mathcal{S}$ systematic uncertainty, shown in Table \ref{tab:unchangedsys}.

 
\begin{table}[htpb]
	\centering
	\caption{ Improved systematic uncertainties of $\mathcal{S}$ expected at 50 ab$^{-1}$. The value in the parenthesis stands for the case that only tag-side resolution is improved and no improvement on $\it{CP}$ side resolution is implemented. }
	\label{tab:reducedsys}
	\begin{tabular}{c| c}
		\hline
		Sources & Improved uncertainty (50 ab$^{-1}$) \\
		\hline
		signal $\Delta t$ shape &  $\sim$0.015($\sim$0.030)\\
		Signal fraction &  $\sim$0.002 \\
		Background $\Delta t$ shape &  $\sim0.001$\\
		wrong tag fraction &  $\sim0.002$\\
		fit bias &  $\sim0.005$\\
		\hline
		Total & $\sim$0.016(0.031)\\
		\hline
	\end{tabular}
\end{table}

\begin{table}[htpb]
	\centering
	\caption{The unchanged systematic uncertainties at 50 ab$^{-1}$ based on the current luminosity of the Belle II data.}
	\label{tab:unchangedsys}
	\begin{tabular}{c| c}
		\hline
		Sources & Unchanged uncertainty (50 ab$^{-1}$) \\
		\hline
		\textit{KsFinder} & $\sim$0.005\\
		physics parameters &  $\sim$0.005 \\
		vertex reconstruction &  $\sim$0.019\\
		tag-side interference &  $\sim0.001$\\
		\hline
		Total & $\sim$0.021\\
		\hline
	\end{tabular}
\end{table}

\subsection{Total uncertainty of $\Delta \mathcal{S}$ at 50 ab$^{-1}$}
The total systematic uncertainty of $\mathcal{S}$ in the 50 ab$^{-1}$ Belle II luminosity is estimated based on the improved sources from Table \ref{tab:reducedsys} and the unchanged sources from Table \ref{tab:unchangedsys}. If $\it{CP}$ side resolution is not improved, the systematic uncertainty is $\sim$0.037. If the $\it{CP}$ side resolution functions is 50\% improved, the systematic uncertainty is reduced to  $\sim0.026$. Both are shown in Table \ref{tab:sys_full}.
\begin{table}[htpb]
	\centering
	\caption{The systematic uncertainty expected in 50 ab$^{-1}$ Belle II luminosity. The first column is the current value of systematic uncertainty of $\mathcal{S}$ in $B^0 \to K_S^0  K_S^0  K_S^0$. The second and third columns are the systematic uncertainties for both $\it{CP}$/tag-side improvements or only tag-side improvement used in the combined estimation.}
	\label{tab:sys_full}
	\begin{tabular}{c| c | c |c}
		\hline
		Luminosity(ab$^{-1}$) & current($\sim$0.063) & $\it{CP}$/tag(50)& only-tag(50)\\
		\hline
		Syst.Uncert.($\mathcal{S}$) & $\sim0.072$ & $\sim$0.026 & $\sim0.037$\\
		\hline
	\end{tabular}
\end{table}


By adding in quadrature using estimated statistical and systematic uncertainties, the total uncertainty for $\mathcal{S}$ in $B^0 \to K_S^0  K_S^0  K_S^0$ in 50 ab$^{-1}$ Belle II luminosity is estimated, as shown in Table \ref{tab:err_full}. Total uncertainty of $\sim 0.048$ or $\sim 0.040$ is expected depending on wether a $\it{CP}$-side resolution is improved or not. Considering that the total uncertainty from $B^0\to J/\psi K_S^0$ at that time is expected to be $\sim 0.005$~\cite{b2book}, $\Delta S$ sensitivity  will be dominated by the total uncertainty in $B^0 \to K_S^0  K_S^0  K_S^0$. The current Belle result from $B^0 \to K_S^0  K_S^0  K_S^0$ on $\Delta S$ is $\sim0.05$ without taking into account any uncertainty, which is close to the theoretical predicted upper-limit. In general, a total uncertainty at about $0.040\sim 0.048$ for $\Delta S$ at Belle II full luminosity is expected to be a much better probe for addressing whether the NP effects in $B^0 \to K_S^0  K_S^0  K_S^0$ exist.

\begin{table}[htpb]
	\centering
	\caption{The total uncertainty of $\mathcal{S}$ in $B^0 \to K_S^0  K_S^0  K_S^0$ expected in 50 ab$^{-1}$ Belle II luminosity, calculated from the expected statistical and systematic uncertainties. The second and third columns are the total uncertainties for both $\it{CP}$/tag-side improvements or only tag-side improvement used in the combined estimation.}
	\label{tab:err_full}
	\begin{tabular}{c|c|c |c}
		\hline
		Luminosity (ab$^{-1}$) & current($\sim$0.063)&$\it{CP}$/tag(50) & only-tag(50)\\
		\hline
		Tot.Ucert.($\mathcal{S}$) & $\sim0.853$ & $\sim0.040$ & $\sim0.048$ \\
		\hline
	\end{tabular}
\end{table}

\subsection{\textit{KsFinder} importance}
While monitoring the uncertainties of the $\it{CP}$ parameters is crucial in searching the hidden NP effects, avoiding bias in the measurement is also critical. If a total uncertainty at $\sim 0.03$ is achieved in future, however, the center value of $\mathcal{S}_{3K_S^0}$ is biased and shifted away from $\mathcal{S}_{J/\psi K_S^0}$, it can lead to a very wrong conclusion about the discovery of the NP effects.
The \textit{KsFinder} contributes to improve the signal purity for measuring $\it{CP}$ parameters, which is essential in controlling the potential bias introduced by the large fraction of background events that yield random $\it{CP}$ asymmetry due to the statistical fluctuation. The larger background events without using \textit{KsFinder} cut in Table \ref{tab:b0select} can produce the wrongly estimated signal fraction ($f_{sig}$) in Equation \ref{eq:tdcpv_all_res} so that the $\it{CP}$ parameters are biased using the biased fit model. Especially when the luminosity is increased in future, the signal fraction uncertainty is expected to be largely reduced, such a biased signal fraction will be treated as a large contribution to the systematic uncertainty compared to the correct ones. 
To demonstrate the effect,  the signal extraction and $\it{CP}$ fit on the 1 ab$^{-1}$ \textit{generic MC} sample without \textit{KsFinder} are performed. In this case, we remove the \textit{KsFinder} cut in Table \ref{tab:b0select} and apply the cut $cosVertexMomentum > 0.9$ which can only achieve $\sim 82\%$ purity for $K_S^0$ in \textit{signal MC}. The signal significance in $M_{bc}$ and $\Delta E$ 2D fit is considerately lower than the ones with using \textit{KsFinder}. The stacked histograms of $M_{bc}$ and $\Delta E$ with much higer background are shown in Figure \ref{fig:hist_2D_highBG} where the red component is signal. There are 352 true signal events and 543 background events inside the signal region by count. In the meanwhile, the 2D fit on $M_{bc}$ and $\Delta E$ are shown in Figure \ref{fig:2Ddata_noks}. From the 2D fit, the signal events number is $389\pm19$ and background events number is $502\pm15$. Clearly the signal fraction defined by the 2D fit in the latter case is biased from the MC truth, which shows a positively biased signal fraction on average, as summarized in Table \ref{tab:ksbias}.

\begin{table}[htpb]
	\centering
	\caption{The number of the signal and background events using \textit{KsFinder} or cut on \textit{cosVertexMomentum} are obtained by the 2D fit, which are compared with the numbers by directly counting the events with $isSignal=1 (0)$.
	It is clear that the number of events obtained by using \textit{KsFinder} is closer to the MC truth.}
	\label{tab:ksbias}
	\begin{tabular}{c| c |c}
		\hline
		Selection & signal  & background \\
		\hline
		${FBDT\_Ks>0.74}$ (fit) & $341\pm20$ & $61\pm17$ \\
		${FBDT\_Ks>0.74}$ (MC) & 336 & 68\\
		${cosVertexMomentum>0.9}$ (fit) & $389\pm19$ & $502\pm15$\\
		${cosVertexMomentum>0.9}$ (MC) & 352 & 543\\
		\hline
	\end{tabular}
\end{table}

From Table \ref{tab:ksbias}, by using \textit{KsFinder}, the true average signal fraction in signal region from 1 ab$^{-1}$ \textit{generic MC} is 83.2\%, and the fit result is $(84.8\pm3.7)\%$. To contrary, by only using ${cosVertexMomentum>0.9}$, the true average signal fraction is 39.3\% and the fit result is $(43.7\pm1.4)\%$, which shows over $3\sigma$ deviation as a strong bias. If such a bias is taken into account as the systematic uncertainty, signal fraction difference is used as a floating value to check the impact on the $\it{CP}$ fit results, which leads to an extra systematic uncertainty from the biased $f_{sig}$ at level of $\sim 0.006$, already larger than any other reducible source except for the signal $\Delta t$ shapes as listed in Table \ref{tab:reducedsys}. Therefore, the development of \textit{KsFinder} is particularly important in the precised $\it{CP}$ measurement for $B^0 \to K_S^0  K_S^0  K_S^0$. The current performance of \textit{KsFinder} is presenting a purity about 95\% in $K_S^0$ reconstruction which means there is still small room for improvements, as well as the background rejection power. The targeted purity and background rejection power of \textit{KsFinder} in future is $\sim 99\%$ on average. The data/MC consistency should also be improved so the current correction ratio $R_{B^0}$ is expected to be reduced to $\sim 1.00\pm 0.01$. Thus the systematic uncertainty from different \textit{KsFinder} responses in between data and MC is assumed to be $\mathcal{O}(0.001)$ as a negligible contribution. 



\begin{figure}[htbp]
	\begin{minipage}[b]{0.5\linewidth}
		\centering 
		\includegraphics[height=6cm]{figures/hist_stacked_generic_mbc_noksfinder.png}
		\label{}
	\end{minipage}
	\begin{minipage}[b]{0.5\linewidth}
		\centering 
		\includegraphics[height=6cm]{figures/hist_stacked_generic_dE_noksfinder.png}
		\label{}
	\end{minipage}
	\caption{$M_{bc}$ and $\Delta E$ stacked histogram of  1 ab$^{-1}$ \textit{generic MC} sample replacing $FBDT\_Ks>0.74$ by ${cosVertexMomentum}>0.9$ in Table \ref{tab:b0select} as a selection criteria, showing a much worse signal significance.}
	\label{fig:hist_2D_highBG}
\end{figure}





\begin{figure}[htbp]
	\begin{minipage}[b]{0.5\linewidth}
		\centering 
		\includegraphics[height=6cm]{figures/mbcfit_noksdata.png}
		\label{}
	\end{minipage}
	\begin{minipage}[b]{0.5\linewidth}
		\centering 
		\includegraphics[height=6cm]{figures/dEfit_noksdata.png}
		\label{}
	\end{minipage}
	\caption{$M_{bc}$ and $\Delta E$ 2D fit on 1 ab$^{-1}$ \textit{generic MC} sample, replacing $FBDT\_Ks>0.74$ by ${cosVertexMomentum}>0.9$ in Table \ref{tab:b0select}. The red is signal component.}
	\label{fig:2Ddata_noks}
\end{figure}

\section{Conclusions.}
In this thesis, we perform the analysis of the $\it{CP}$ parameter measurement in $B^0 \to K_S^0  K_S^0  K_S^0$ using the early Belle II data, which is targeted to search for the NP effects in the penguin dominated $b\to s$ transition. The reconstruction of $K_S^0$ is initially done by using a traditional cut-based method with a large fraction of fake candidates. Thus a new MVA-based \textit{KsFinder} is developed by using FastBDT algorithm  which can effectively improve the signal purity. For the reconstruction of $B^0$, we take advantage of two variable $M_{bc}$ and $\Delta E$ to select the events with a sufficient continuum background suppression. To obtain the signal fraction for each $B^0$ events, a 2D fit model is established and fitted using 62.8 fb$^{-1}$ data. 
The model of the resolution of vertex positions has been studied using MC sample and sideband data based on the understanding of vertex reconstruction performance in the current Belle II detectors. To make a proper use of the reconstructed vertex information and perform a $\it{CP}$ fit compactly, a new $\it{CP}$ fitter is built and being validated, which will serve as a multi-functional analysis tool for the Belle II $\it{CP}$ violation study in future. 
The $\it{CP}$ parameter measurement is performed based on the validation of analysis strategies by the blind analysis that shows a consistent result for $\it{CP}$ parameters compared to the simulation input. The linearity and pull of the $\it{CP}$ fit are checked to demonstrate the reliability of the fit procedures. The fit result on $B^0$ lifetime using the experiment data is also agreed with the current value in PDG with a relatively large statistical uncertainty due to the low statistics of data.

After the $\it{CP}$ fit procedures are validated, the $\it{CP}$ parameters $\mathcal{S}$ and $\mathcal{A}$ using 62.8 ab$^{-1}$ Belle II early data in 2019 and 2020 spring and summer are obtained. The result is

\begin{equation}\label{eq:data_fit_cp}
\begin{split}
\mathcal{S}=- \text{sin}(2\phi_1) & = -0.82 \pm 0.85~(\text{stat}) \pm 0.07~(\text{syst})~, \\
\mathcal{A} & = -0.21\pm 0.28~(\text{stat}) \pm 0.06~(\text{syst})~.\\
\end{split}
\end{equation}  

The result agrees with the prediction of the Standard Model and the previous results from Belle~\cite{kang2020measurement} and BaBar~\cite{lees2012amplitude}. The measurement precision of $\it{CP}$ parameters in this thesis is majorly limited by the large statistical uncertainty.


\section{Prospect}
Even though the current result on $\it{CP}$ parameters are dominated by the large uncertainty, the previous discussions about the future results have shown a good potential of searching for the NP effects in $B^0 \to K_S^0  K_S^0  K_S^0$ based on the current analysis in this thesis. At integral luminosity at 50 ab$^{-1}$, the uncertainty on $\mathcal{S}$ would be reduced to a comparable value around $0.040\sim0.048$ realistically, as shown in Figure \ref{fig:tot_exp}, where the statistical and reducible systematic uncertainties are assumed to be scaled by the squared root of the integrated luminosity. The expected sensitivity in full Belle II data is proven to be competitive and the analysis workflow is built which will be further improved along with the future Belle II data taking and MC production. The progress that has been made in this thesis paves a well-constructed and solid path for searching the NP effects in time dependent $\it{CP}$ violation study of $B^0 \to K_S^0  K_S^0  K_S^0$ at the full Belle II luminosity in future.
 
\begin{figure}[htpb]
	\centering
	\includegraphics[width=0.7\linewidth]{tot_unc_1}
	\includegraphics[width=0.7\linewidth]{tot_unc_2}
	\caption{The expected total uncertainty of $\mathcal{S}$ in $B^0 \to K_S^0  K_S^0  K_S^0$, where the dashed lines are the statistical(black), $\it{CP}$/tag-side improved systematic(green) and only tag-side improved systematic(red) uncertainties, with the corresponding solid lines as the total uncertainties. The top is the overview for the whole Belle II luminosity range from now, and the bottom is $y$-axis zoom-in.}
	\label{fig:tot_exp}
\end{figure}

